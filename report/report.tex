\documentclass{article}
\usepackage[utf8]{inputenc}
\usepackage{polski}
\usepackage{amssymb}
\usepackage{amsmath,amsthm}
\usepackage{amsfonts}
\usepackage{graphicx}
\usepackage{textcomp}
\usepackage{gensymb}
\usepackage{tikz}
\usepackage{geometry}
\usepackage{subcaption}
\usepackage{graphicx}
\usepackage{wrapfig}
\usepackage{hyperref}
\hypersetup{
    colorlinks=false,
    linktoc=all,
}
\newgeometry{tmargin=3cm, bmargin=3cm, lmargin=3cm, rmargin=3cm}

\title{Secure Card Game}
\author{Jan Klimczak \\ Filip Konieczny \\ Rafał Pyzik}
\date{24.06.2024}

\begin{document}

\maketitle

\section{Opis projektu}

Celem projektu jest implementacja podstawowych funkcjonalności związanych ze wszelkiego rodzaju grami z ustaloną talią kart, takimi jak
\begin{enumerate}
    \item Tasowanie talii kart
    \item Rozdawanie kart
    \item Branie karty ze stosu
    \item Zagranie karty
\end{enumerate}

Oczywiście chcemy, aby wszystkie funkcjonalności były kryptograficznie bezpieczne, co, między innymi,  implikuje, że nikt nie może ustalić kolejności kart w talii oraz by nikt nie informacji ponad to, co wydarzyło się w grze i własnych kart.  

Powyższe funkcjonalności zaprezentowane są na przykładzie prostej gry przypominającej wojnę. Przebieg rozgrywki jest następujący:

\begin{enumerate}
    \item Talia jest tasowana
    \item Każdy z graczy ciągnie z talii 5 kart
    \item Gracze kolejno rzucają karty na stół
    \item Wygrywa karta, która ma najwyższą wartość wśród wszystkich kart w kolorze pierwszej rzuconej karty (wartości kart są standardowe 2, 3, \ldots, 10, J, Q, K, A)
    \item Gracz którego karta wygrała zdobywa punkt i będzie zaczynał następną turę
    \item Jeśli stos kart jest niepusty, każdy ciągnie kartę, zaczynając od osoby, którego karta wygrała
\end{enumerate}


\section{Kryptografia}

Karty są reprezentowane jako punkty na krzywej eliptycznej. Niech $G$ to generator grupy na której pracujemy. Wtedy karty to $C = \{G,2G,\ldots,52G\}$, gdzie stosujemy notację addytywną. W wyniku procesu tasowania każdy z graczy zaszyfruje karty swoim kluczem prywatnym, w wyniku czego karta to $x_1x_2\ldots c$, gdzie $c \in C$, a $x_i$ to klucze prywatne graczy (będące skalarami).

Oprócz standardowego założenia, że CDH jest trudny na naszej krzywej eliptycznej, dodatkowo zakładamy to także o decyzyjnej wersji. W szczególności, użyta w protokole krzywa eliptyczna nie może wspierać dwuliniowego parowania.

\subsection{Protokoły}
\subsubsection{Tasowanie}

Tasowanie to najbardziej skomplikowany z użytych protokołów. Wysokopoziomowo, każdy z graczy aplikuje do talii wybraną przez siebie permutację. Widzimy, że jeśli gracz nie jest w stanie poznać permutacji innego gracza, to gracz nie jest w stanie poznać końcowej permutacji talii (lepiej niż zgadując).


Potasowana talia to ciąg $(x_1^1x_2^1\ldots c_1, x_1^2,x_2^2\ldots c_2, \ldots, x_1^kx_2^k\ldots c_k)$.

Zaimplementowaliśmy dwa protokoły. Zacznijmy od opisu prostszego:

\begin{enumerate}
    \item Gracz $A$ wybiera klucz prywatny $A$ i zaszyfrowuje każdą kartę za pomocą $A$, a następnie tasuje talię. Każda karta jest postaci $Ac$, gdzie $c \in C$,

    \item Kolejni gracze robią to samo,

    \item Gracz $A$ wybiera $52$ klucze prywatne. Następnie ściąga szyfrowanie kluczem $A$ i szyfruje kolejne karty kolejnymi kluczami. Każda karta jest postaci $x_1^iB\ldots c_i$,

    \item Kolejni gracze robią to samo.
\end{enumerate}

Na końcu każdy z graczy ma ciąg $(x_1^1x_2^1\ldots c_1, x_1^2,x_2^2\ldots c_2, \ldots, x_1^kx_2^k\ldots c_k)$, gdzie $c_i$ jest permutacją zbioru $C$.

Protokół ten wymaga założenia o trudności DDH ze względu na pierwszy krok. Gracz drugi otrzymuje $Ai\cdot G$ dla $i\in \{1,\ldots,52\}$, więc w szczególności wśród otrzymanych kart są $AG$ oraz $A2G$. Może on przetestować wszystkie pary kart $x,y$ czy trójka $(x,2G,y)$ jest trójką DH, co daje mu kandydatów na odpowiednie karty. W podobny sposób może on odtworzyć całą permutację zastosowaną przez poprzednią osobę.

Jednak nawet z założeniem DDH protokół ma parę problemów. Gracz może łatwo oszukać w kroku 3 duplikując karty. Zamiast zaszyfrować swoimi kluczami otrzymane karty, to zastępuje niektóre karty z talii inną otrzymaną kartą (np. zamiast zwrócić $x_1a,x_2b,x_3c$ to zwraca $x_1a,x_2a,x_3c$).

Jeśli zrobi to gracz pierwszy, to protokół się nie zakończy się błędem: inni gracze nie zauważą ingerencji, ich klucze prywatne z kroku 2 będą wciąż działać jak powinny, podobnie jak klucze z kroku 4 (bo jeszcze ich nie zaaplikowali). Zaskutkuje to tym, że w pewnym momencie w trakcie gry jedna karta może pojawić się wielokrotnie.

W celu naprawienia tej luki, na końcu gry wszyscy ujawniają swoje klucze i deszyfrowana jest cała talia by upewnić się, że jest permutacją zbioru $C$.

Aby poprawić wspomniane aspekty, druga wersja protokołu narzuca na graczy dostarczenie zero\-knowledge dowodów, że poprawnie wykonali wszystkie kroki.

Będziemy potrzebować trzech pomocniczych podprotokołów będących wariacjami podpisu Schnorra. Poniżej stosujemy konwencję, gdzie $Hash(\ldots || X_i || \ldots) = Hash(\ldots || X_1 || \ldots || X_n || \ldots)$.
\\

\noindent \textbf{Masking} Gracz ma dane publiczny punkt $P$ oraz prywatny klucz $x$. Publikuje on $P'$ i chce udowodnić $P' = xP$.
\\

\noindent Losuje on $r$ i liczy $R = rP$, $e = Hash(R || P || P')$ (challenge) i $s = r - xe$ (dowód). Publikuje $(e,s)$. 
\\

\noindent Weryfikacja polega na policzeniu $R^v = sP + eP'$ i sprawdzeniu czy $e = Hash(R^v || P || P')$. Warto zauważyć, że jest to w zasadzie dokładnie podpis Schnorra.
\\

\noindent \textbf{Multimasking}
Gracz ma dane publiczne punkty $(P_1,P_2,\ldots,P_n)$ oraz prywatny klucz $x$. Publikuje $(P_1',P_2',\ldots,P_n')$ i chce udowodnić, że $P_i'=xP_i$.
\\

\noindent Losuje on $r$ i liczy $R_i = rP_i$, $e = Hash(R_i^v || P_i || P_i')$ (challenge) i $s = r - xe$ (dowód). Publikuje $(e,s)$.
\\

\noindent Weryfikacja polega na policzeniu $R_i^v = sP_i + eP_i'$ i sprawdzeniu czy $e = Hash(R_i^v || P_i || P_i')$.
\\


\noindent \textbf{1-in-2} Gracz ma dane punkty $A,B$ oraz prywatny klucz $x$. Publikuje on $P$ i chce udowodnić, że $P = xA$ lub $P=xB$, bez ujawiania, który przypadek zachodzi.
\\

\noindent Opiszemy protokół dla przypadku $P = xA$.
\noindent \begin{enumerate}
    \item Losuje $r$ i liczy $R_a = rA$, $e_b = Hash(R_a || P || A || B)$,
    \item Losuje $s_b$ i liczy $R_b = e_bP + s_bB$,
    \item Liczy $e_a = Hash(R_b || P || A || B)$,
    \item Liczy $s_a = r - e_ax$,
    \item Zwraca $(e_a,e_b,s_a,s_b)$.
\end{enumerate}

\noindent Weryfikacja polega na policzeniu $R_a^v = s_aA + e_aP$ oraz $R_b^v = s_bB + e_bP$ i sprawdzeniu czy $e_a = Hash(R_b^v || P || A || B)$ oraz $e_b = Hash(R_a^v || P || A || B)$.
\\

\noindent \textbf{1-in-2} Gracz ma dane punkty $A,B$ oraz prywatny klucz $x$. Publikuje on $(C,D)$ i chce udowodnić, że $(C,D) = (xA,xB)$ lub $(C,D) = (xB,xA)$ bez ujawiania, który przypadek zachodzi.
\\

\noindent Wystarczy dwukrotnie zastosować protokół 1-in-2.
\\


Korzystając z protokołów swap oraz multimaskingu można udowodnić, że rzeczywiście pooprawnie wykonało się krok 1 poprzedniego protokołu. Można to osiągnąć poprzez użycie sieci sortującej na odwrót tzn. sieć sortująca jest w stanie posortować dowolną permutację za pomocą statycznego ciągu komparatorów. Przechodząc przez nią na odwrót i używając protokołu swap dla każdego komparatora jesteśmy w stanie dotrzeć do dowolnej permutacji.


Takie podejście daje algorytm wysyłający $\log{n}^2$ razy liniowo wiele punktów i skalarów. Dla $n = 52$ logarytm z $n$ jest względnie duży ($\log{n}^2 > n$). Nawet użycie najlepszej znanej sieci sortującej dla $n=52$ daje nam 19 warstw sieci, co daje prawie tysiąc skalarów i punktów do wysłania. Zamiast tego zdecydowaliśmy się użyć rozszerzonej wersji protokołu 1-in-2, mianowicie 1-in-$n$, którego $n$-krotna aplikacja (jak w swap) pozwoli nam udowodnić cały shuffle za jednym zamachem wysyłając liniowo wiele punktów i kwadratowo wiele skalarów.
\\

\noindent \textbf{1-in-$n$} Gracz ma dane punkty $P_1,P_2,\ldots,P_n$ oraz prywatny klucz $x$. Publikuje on $P$ i chce udowodnić, że $P = xP_i$ dla pewnego $i$, bez ujawiania wybranego $i$.
\\

\noindent Opiszemy protokół dla $P = xP_1$
\noindent \begin{enumerate}
    \item Losuje $r$ i liczy $R_1 = rP_1, e_2 = Hash(R_1 || P || P_i)$,
    \item Losuje $s_2$ i liczy $R_2 = e_2P + s_2P_2$,
    \item Liczy $e_3 = Hash(R_2 || P || P_i)$,
    \item Losuje $s_3$ i liczy $R_3 = e_3P + s_3P_3$,
    \item $\ldots$
    \item Liczy $e_1 = Hash(R_n || P || P_i)$,
    \item Liczy $s_1 = r - e_1x$,
    \item Zwraca $(e_1,\ldots,e_n,s_1,\ldots,s_n)$,
\end{enumerate}

\noindent Weryfikacja polega na policzeniu $R_i^v = e_iP + s_iP_i$ i sprawdzeniu $e_i = Hash(R_i^v || P || P_j)$.

\subsubsection{Ciągnięcie z talii}

Gracz ogłasza, że chce pociągnąć kartę z talii. Reszta weryfikuje, że ma on do tego prawo, a następnie publikuje swoje klucze prywatne związane z kartą na szczycie stosu. Gracz następnie może użyć wspomniane klucze prywatne oraz swój własny do poznania wartości karty.

Rozdawanie kart to wielokrotne zaaplikowanie tego protokołu.

\subsubsection{Zagrywanie karty}

Gracz ogłasza, którą kartę chce zagrać, a następnie publikuje swój klucz prywatny. Pozostałe klucze zostały już ujawnione, więc tożsamość karty staje się publiczna.

\section{Implementacja i struktura projektu}

Do części kryptograficznej użyliśmy implementacji krzywych eliptycznych z crate'a z rodziny \texttt{arkworks}, a konkretnie krzywej Pallas. Poza tym protokoły zaimplementowane zostały ,,ręcznie''.

Rozgrywka jak i protokoły zaimplementowane zostały jako automat skończony.

GUI zaimplementowane zostało we framework'u \texttt{gtk}.




\end{document}
